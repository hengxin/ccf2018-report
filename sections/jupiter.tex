% file: sections/jupiter.tex

\section{Jupiter}

%%%%%%%%%%%%%%%%%%%%
\begin{frame}{}
  \begin{center}
    \begin{mdframed}[frametitle = {\large Brief Announcement}, frametitlerule = true, frametitlebackgroundcolor = brown!20,
      frametitleaboveskip = 8pt, frametitlebelowskip = 8pt, innertopmargin = 10pt]
      {\Large The \blue{Jupiter protocol}~\ncite{Nichols:UIST95}\footfullcite{Nichols:UIST95} for replicated list \\
      \red{satisfies} the \blue{weak list specification}~\ncite{Attiya:PODC16}\footfullcite{Attiya:PODC16}.} \\[15pt]
    \end{mdframed}

    \vspace{0.20cm}
    {\large This was proposed as a conjecture in a PODC paper~\ncite{Attiya:PODC16}.}
  \end{center}
\end{frame}
%%%%%%%%%%%%%%%%%%%%

%%%%%%%%%%%%%%%%%%%%
\begin{frame}{}
  \centerline{\Huge \teal{Replicated List}}
\end{frame}
%%%%%%%%%%%%%%%%%%%%

%%%%%%%%%%%%%%%%%%%%
\begin{frame}{}
  \centerline{\teal{\Large \red{Replicated} collaborative text editing systems}}

  \fignocaption{width = 0.55\textwidth}{figs/coeditor}

  \pause
  \vspace{-0.50cm}
  \begin{center}
    Replicas are required to respond to user operations \red{immediately}.  \\[3pt]

    Updates are propagated to other replicas \red{asynchronously}.
  \end{center}
\end{frame}
%%%%%%%%%%%%%%%%%%%%

%%%%%%%%%%%%%%%%%%%%
\begin{frame}{}
  \centerline{\Large \red{Replicated list object}: to model the core functionality}
  \vspace{0.30cm}

  \begin{center}
    \begin{minipage}{0.70\textwidth}
      \begin{description}
	\setlength{\itemsep}{10pt}
	\item[$\textsc{Ins}(a, p):$] Insert $a$ at position $p$.
	\item[$\textsc{Del}(p):$] Delete the element at position $p$.
	\item[$\textsc{Read}:$] Return the list.
      \end{description}
    \end{minipage}
  \end{center}
\end{frame}
%%%%%%%%%%%%%%%%%%%%

%%%%%%%%%%%%%%%%%%%%
\begin{frame}{}
  \fignocaption{width = 0.80\textwidth, frame}{figs/podc16-attiya}

  \vspace{0.20cm}
  \begin{definition}[Weak List Specification \wlspec{}~\ncite{Attiya:PODC16}]
    Informally, \wlspec{} requires the ordering between \red{elements that are not deleted} to be consistent across the system.
  \end{definition}

  \vspace{0.60cm}
  \centerline{\teal{\large Specify a global property on all states across the system.}}
\end{frame}
%%%%%%%%%%%%%%%%%%%%

%%%%%%%%%%%%%%%%%%%%
\begin{frame}{}
  \centerline{\blue{\Large Pairwise state compatibility property:}}

  \vspace{0.40cm}
  \begin{quote}
    For any \red{pair} of list states, there \red{cannot} be two elements $a$ and $b$ such that $a$ precedes $b$ in one state
    but $b$ precedes $a$ in the other.
  \end{quote}

  \vspace{0.30cm}
  \begin{columns}
    \column{0.50\textwidth}
      \fignocaption{width = 0.45\textwidth}{figs/ex-weak-list-spec}
      \vspace{-0.60cm}
      \fignocaption{width = 0.30\textwidth}{figs/red-cross}
    \column{0.50\textwidth}
      \fignocaption{width = 0.55\textwidth}{figs/ex-strong-list-spec}
      \vspace{-0.60cm}
      \fignocaption{width = 0.20\textwidth}{figs/green-check}
  \end{columns}
\end{frame}
%%%%%%%%%%%%%%%%%%%%
%%%%%%%%%%%%%%%%%%%%
\begin{frame}{}
  \centerline{\large Jupiter adopts the \red{client-server} architecture~\ncite{Nichols:UIST95}:}

  \begin{center}
    \begin{minipage}{0.50\textwidth}
      \input{tikz/jupiter-cs-tikz}
    \end{minipage}
  \end{center}

  \vspace{-1.00cm}
  \uncover<2->{
    \begin{center}
      Operations are \red{totally ordered} at the server {\footnotesize (replica)}. \\[6pt]
      Client {\footnotesize (replica)} $\xrightarrow[]{\quad \text{FIFO} \quad}$ server $\xrightarrow[]{\quad \text{FIFO} \quad}$ other clients
    \end{center}
  }
\end{frame}
%%%%%%%%%%%%%%%%%%%%

%%%%%%%%%%%%%%%%%%%%
\begin{frame}{}
  \begin{center}
    {\large To achieve convergence, Jupiter uses \red{$2D$ state spaces}~\ncite{Xu:CSCW14}
    to manage how and when to perform \red{OTs}~\footnote{OT: Operational Transformation}~\ncite{Ellis:SIGMOD89}.}
  \end{center}

  \fignocaption{width = 0.32\textwidth}{figs/2d-statespace}

  \begin{center} 
    There can be \red{$\le 2$ edges} coming from the same node ({\textsc{\footnotesize Local}} or \textsc{\footnotesize Global}).
  \end{center}
\end{frame}
%%%%%%%%%%%%%%%%%%%%

%%%%%%%%%%%%%%%%%%%%
\begin{frame}{}
  \centerline{\large Each \red{client} maintains a $2D$ state space.}

  % \fignocaption{width = 0.50\textwidth}{figs/jupiter-illustration-client3-notations}
  \fignocaption{width = 0.60\textwidth}{figs/jupiter-illustration}

  \centerline{\large The \red{server} maintains $n \; (=3)$ $2D$ state spaces, one for each client.}
\end{frame}
%%%%%%%%%%%%%%%%%%%%

%%%%%%%%%%%%%%%%%%%%
\begin{frame}{}
  \begin{center}
    {\large \red{Global property} on all replica states specified by \blue{\wlspec{}}}

    \vspace{0.20cm}
    \fignocaption{width = 0.30\textwidth}{figs/mismatch}
    \vspace{0.20cm}

    {\large \red{Local view} each replica maintains in \blue{Jupiter}}
  \end{center}
\end{frame}
%%%%%%%%%%%%%%%%%%%%

%%%%%%%%%%%%%%%%%%%%
\begin{frame}{}
  \begin{center}
    {\large CJupiter maintains an \red{$n$-ary ordered state space} for each replica.}
  \end{center}

  \fignocaption{width = 0.40\textwidth}{figs/cjupiter-allinone}

  \begin{center} 
    There can be \red{more than two edges} coming from the same node. \\[5pt]
    Edges from the same node are \red{totally ordered} by associated operations.
  \end{center}
\end{frame}
%%%%%%%%%%%%%%%%%%%%

%%%%%%%%%%%%%%%%%%%%
\begin{frame}{}
  \begin{center}
    \begin{prop}[Compactness of CJupiter (Informal)]
      {\large At a high level, CJupiter maintains only \red{one} $n$-ary ordered state space.}
    \end{prop}

    \vspace{0.20cm}
    % \resizebox{0.50\textwidth}{!}{\input{tikz/cjupiter-allinone-path}}
    \fignocaption{width = 0.40\textwidth}{figs/cjupiter-allinone-path}

    \vspace{0.20cm}
    Each replica behavior corresponds to a \red{path} going through this state space.
  \end{center}
\end{frame}
%%%%%%%%%%%%%%%%%%%%

%%%%%%%%%%%%%%%%%%%%
\begin{frame}{}
  \begin{Theorem}[Equivalence of CJupiter and Jupiter]
    Under the same schedule, the behaviors of corresponding replicas in CJupiter and Jupiter are the same.
  \end{Theorem}

  \vspace{1.0cm}
  \centerline{\large From the perspectives of both the server and the clients.}
\end{frame}
%%%%%%%%%%%%%%%%%%%%
%%%%%%%%%%%%%%%%%%%%
\begin{frame}{}
  \begin{center}
    {\large We focus on a single $n$-ary ordered state space.}
  \end{center}

  \fignocaption{width = 0.45\textwidth}{figs/cjupiter-allinone-path}

  \begin{center}
    {\large We show the \red{pairwise state compatibility} property \blue{in three steps}.}
  \end{center}
\end{frame}
%%%%%%%%%%%%%%%%%%%%

%%%%%%%%%%%%%%%%%%%%
\begin{frame}{}
  \centerline{\circled{1} Take any two nodes/states $v_1$ and $v_2$.}

  \begin{lemma}[LCA (Lowest Common Ancestor)]
    Each pair of states in the $n$-ary ordered state space has a \red{unique} LCA.
  \end{lemma}

  \begin{columns}
    \column{0.60\textwidth}
      \fignocaption{width = 0.50\textwidth}{figs/lca}
      \column{0.30\textwidth}
	\[
	  v_0 = \text{LCA}(v_1, v_2)
	\]
  \end{columns}
\end{frame}
%%%%%%%%%%%%%%%%%%%%

%%%%%%%%%%%%%%%%%%%%
\begin{frame}{}
  \centerline{\circled{2} Consider the paths to $v_1$ and $v_2$ from their LCA $v_0$.}

  \begin{lemma}[Disjoint Paths]
    The set of operations $O_{v_0 \leadsto v_1}$ along $P_{v_0 \leadsto v_1}$ 
    is \red{disjoint} from the set of operations $O_{v_0 \leadsto v_2}$ along $P_{v_0 \leadsto v_2}$.
  \end{lemma}

  \begin{columns}
    \column{0.60\textwidth}
	\fignocaption{width = 0.50\textwidth}{figs/disjoint-paths}
      \column{0.30\textwidth}
	\[
	  v_0 = \text{LCA}(v_1, v_2)
	\]
  \end{columns}
\end{frame}
%%%%%%%%%%%%%%%%%%%%

%%%%%%%%%%%%%%%%%%%%
\begin{frame}{}
  \centerline{\circled{3} Consider the states in these two paths.}

  \begin{lemma}[Compatible Paths]
    Each pair of states consisting of one state $v$ in $P_{v_0 \leadsto v_1}$ and the other $v'$ in $P_{v_0 \leadsto v_2}$ are \red{compatible}.
  \end{lemma}

  \begin{columns}
    \column{0.60\textwidth}
	\fignocaption{width = 0.55\textwidth}{figs/compatible-paths}
      \column{0.40\textwidth}
	\[
	  v_0 = \text{LCA}(v_1, v_2)
	\]

	\vspace{0.50cm}
	\begin{center}
	  {In particular, \\ $v_1$ and $v_2$ are compatible.}
	\end{center}
  \end{columns}
\end{frame}
%%%%%%%%%%%%%%%%%%%%


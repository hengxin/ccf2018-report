% file: sections/jupiter.tex

\section{Jupiter}

%%%%%%%%%%%%%%%%%%%%
\begin{frame}{}
  \begin{center}
    \begin{mdframed}[frametitle = {\large Brief Announcement}, frametitlerule = true, frametitlebackgroundcolor = brown!20,
      frametitleaboveskip = 8pt, frametitlebelowskip = 8pt, innertopmargin = 10pt]
      {\Large The \blue{Jupiter protocol}~\ncite{Nichols:UIST95}\footfullcite{Nichols:UIST95} for replicated list \\
      \red{satisfies} the \blue{weak list specification}~\ncite{Attiya:PODC16}\footfullcite{Attiya:PODC16}.} \\[15pt]
    \end{mdframed}

    \vspace{0.20cm}
    {\large This was proposed as a conjecture in a PODC paper~\ncite{Attiya:PODC16}.}
  \end{center}
\end{frame}
%%%%%%%%%%%%%%%%%%%%

%%%%%%%%%%%%%%%%%%%%
\begin{frame}{}
  \centerline{\Huge \teal{Replicated List}}
\end{frame}
%%%%%%%%%%%%%%%%%%%%

%%%%%%%%%%%%%%%%%%%%
\begin{frame}{}
  \centerline{\teal{\Large \red{Replicated} collaborative text editing systems}}

  \fignocaption{width = 0.55\textwidth}{figs/coeditor}

  \pause
  \vspace{-0.50cm}
  \begin{center}
    Replicas are required to respond to user operations \red{immediately}.  \\[3pt]

    Updates are propagated to other replicas \red{asynchronously}.
  \end{center}
\end{frame}
%%%%%%%%%%%%%%%%%%%%

%%%%%%%%%%%%%%%%%%%%
\begin{frame}{}
  \centerline{\Large \red{Replicated list object}: to model the core functionality}
  \vspace{0.30cm}

  \begin{center}
    \begin{minipage}{0.70\textwidth}
      \begin{description}
	\setlength{\itemsep}{10pt}
	\item[$\textsc{Ins}(a, p):$] Insert $a$ at position $p$.
	\item[$\textsc{Del}(p):$] Delete the element at position $p$.
	\item[$\textsc{Read}:$] Return the list.
      \end{description}
    \end{minipage}
  \end{center}
\end{frame}
%%%%%%%%%%%%%%%%%%%%

%%%%%%%%%%%%%%%%%%%%
\begin{frame}{}
  \fignocaption{width = 0.80\textwidth, frame}{figs/podc16-attiya}

  \vspace{0.20cm}
  \begin{definition}[Weak List Specification \wlspec{}~\ncite{Attiya:PODC16}]
    Informally, \wlspec{} requires the ordering between \red{elements that are not deleted} to be consistent across the system.
  \end{definition}

  \vspace{0.60cm}
  \centerline{\teal{\large Specify a global property on all states across the system.}}
\end{frame}
%%%%%%%%%%%%%%%%%%%%

%%%%%%%%%%%%%%%%%%%%
\begin{frame}{}
  \centerline{\blue{\Large Pairwise state compatibility property:}}

  \vspace{0.40cm}
  \begin{quote}
    For any \red{pair} of list states, there \red{cannot} be two elements $a$ and $b$ such that $a$ precedes $b$ in one state
    but $b$ precedes $a$ in the other.
  \end{quote}

  \vspace{0.30cm}
  \begin{columns}
    \column{0.50\textwidth}
      \fignocaption{width = 0.45\textwidth}{figs/ex-weak-list-spec}
      \vspace{-0.60cm}
      \fignocaption{width = 0.30\textwidth}{figs/red-cross}
    \column{0.50\textwidth}
      \fignocaption{width = 0.55\textwidth}{figs/ex-strong-list-spec}
      \vspace{-0.60cm}
      \fignocaption{width = 0.20\textwidth}{figs/green-check}
  \end{columns}
\end{frame}
%%%%%%%%%%%%%%%%%%%%
%%%%%%%%%%%%%%%%%%%%
\begin{frame}{}
  \centerline{\large Jupiter adopts the \red{client-server} architecture~\ncite{Nichols:UIST95}:}

  \begin{center}
    \begin{minipage}{0.50\textwidth}
      % file: tikz/jupiter-cs-tikz.tex

\def\s{s}  % server
\def\b{bot} % literal string
\tikzset{seq/.style = {draw, circle, outer sep = 5pt, inner sep = 2pt, scale = #1, left}}

% send: the sender sends operation to the receiver
\newcommand{\send}[5]{% #1: sender; #2: receiver; #3: sender pos; #4: receiver pos; #5: seq. number;
  \draw[->]  ($(#1)!#3!(#1\b)$) node[seq = 0.40] {#5} to ($(#2)!#4!(#2\b)$) node[seq = 0.60] {$#5$};
}

% csend: client sends operation to the server
\newcommand{\csend}[6]{% #1: client; #2: client pos; #3: server pos; #4: seq. number; #5: client state; #6: server state
  \draw[->]  ($(#1)!#2!(#1\b)$) node[seq = 0.60, draw = none] (#4) {} to ($(\s)!#3!(\s\b)$) {};
}

% ssend: the server sends operation to client 
\newcommand{\ssend}[5]{% #1: client; #2: server pos; #3: client pos; #4: seq. number; #5: client state
  \draw[->, dashed]  ($(\s)!#2!(\s\b)$) to ($(#1)!#3!(#1\b)$) node[solid, seq = 0.60] {$#4$};
}

\newcommand{\ins}[2]{$\textcolor{blue}{\textsc{Ins}(#1,#2)}$}
\newcommand{\del}[2]{$\textcolor{blue}{\textsc{Del}(#2)}$}  % del(a,p): ignoring the element 'a'

\begin{tikzpicture}[
    timeline/.style = {very thick}, >=Stealth, 
    op/.style = {font = \small, above left = -0.20cm and -0.40cm of #1, sloped},
    replica/.style = {align = center}]
  \uncover<1->{
    \node[replica] (\s) {\textcolor{brown}{$S$}};
    \node[replica, right = of s] (c1) {$\textcolor{brown}{C_1}$};
    \node[replica, right = of c1] (c2) {$\textcolor{brown}{C_2}$};
    \node[replica, right = of c2] (c3) {$\textcolor{brown}{C_3}$};

    \foreach \r/\rbot in {s/sbot, c1/c1bot, c2/c2bot, c3/c3bot} {
	  \node[below = 5.0cm of \r] (\rbot) {};
	  \draw[timeline] (\r) to (\rbot);
    }
    \draw[timeline, ultra thick, red] (s) to (sbot);
  }

  \uncover<2->{
    % Clients send messages (ordered by positions at server) to the server
    \csend{c1}{0.15}{0.20}{1}{x}{x}
    \csend{c1}{0.35}{0.40}{2}{}{}
    \csend{c2}{0.45}{0.55}{3}{ax}{a}
    \csend{c3}{0.50}{0.75}{4}{xb}{}
  }

  \uncover<2->{
    % Operations are totally ordered at the server.
    \foreach \num/\pos in {1/0.20, 2/0.40, 3/0.55, 4/0.75} {
      \node [seq = 0.80, fill = red!40] at ($(\s)!\pos!(\s\b)$) {$\num$};
    }
  }

  % \node (o1) [op = {1}, rotate = 7] {};
  % \node (o2) [op = {2}, rotate = 7] {};
  % \node (o3) [op = {3}, rotate = 8] {};
  % \node (o4) [op = {4}, rotate = 15] {};

  \uncover<2->{
    % The server sends messages (ordered by positions at server) to clients.
    \ssend{c2}{0.40}{0.65}{2}{a}
    \ssend{c3}{0.40}{0.70}{2}{b}

    \ssend{c2}{0.20}{0.30}{1}{x}
    \ssend{c3}{0.20}{0.25}{1}{x}


    \ssend{c1}{0.55}{0.60}{3}{a}
    \ssend{c3}{0.55}{0.90}{3}{}

    \ssend{c1}{0.75}{0.90}{4}{}
    \ssend{c2}{0.75}{0.90}{4}{}
  }
\end{tikzpicture}

    \end{minipage}
  \end{center}

  \vspace{-1.00cm}
  \uncover<2->{
    \begin{center}
      Operations are \red{totally ordered} at the server {\footnotesize (replica)}. \\[6pt]
      Client {\footnotesize (replica)} $\xrightarrow[]{\quad \text{FIFO} \quad}$ server $\xrightarrow[]{\quad \text{FIFO} \quad}$ other clients
    \end{center}
  }
\end{frame}
%%%%%%%%%%%%%%%%%%%%

%%%%%%%%%%%%%%%%%%%%
\begin{frame}{}
  \begin{center}
    {\large To achieve convergence, Jupiter uses \red{$2D$ state spaces}~\ncite{Xu:CSCW14}
    to manage how and when to perform \red{OTs}~\footnote{OT: Operational Transformation}~\ncite{Ellis:SIGMOD89}.}
  \end{center}

  \fignocaption{width = 0.32\textwidth}{figs/2d-statespace}

  \begin{center} 
    There can be \red{$\le 2$ edges} coming from the same node ({\textsc{\footnotesize Local}} or \textsc{\footnotesize Global}).
  \end{center}
\end{frame}
%%%%%%%%%%%%%%%%%%%%

%%%%%%%%%%%%%%%%%%%%
\begin{frame}{}
  \centerline{\large Each \red{client} maintains a $2D$ state space.}

  % \fignocaption{width = 0.50\textwidth}{figs/jupiter-illustration-client3-notations}
  \fignocaption{width = 0.60\textwidth}{figs/jupiter-illustration}

  \centerline{\large The \red{server} maintains $n \; (=3)$ $2D$ state spaces, one for each client.}
\end{frame}
%%%%%%%%%%%%%%%%%%%%

%%%%%%%%%%%%%%%%%%%%
\begin{frame}{}
  \begin{center}
    {\large \red{Global property} on all replica states specified by \blue{\wlspec{}}}

    \vspace{0.20cm}
    \fignocaption{width = 0.30\textwidth}{figs/mismatch}
    \vspace{0.20cm}

    {\large \red{Local view} each replica maintains in \blue{Jupiter}}
  \end{center}
\end{frame}
%%%%%%%%%%%%%%%%%%%%

%%%%%%%%%%%%%%%%%%%%
\begin{frame}{}
  \begin{center}
    {\large CJupiter maintains an \red{$n$-ary ordered state space} for each replica.}
  \end{center}

  \fignocaption{width = 0.40\textwidth}{figs/cjupiter-allinone}

  \begin{center} 
    There can be \red{more than two edges} coming from the same node. \\[5pt]
    Edges from the same node are \red{totally ordered} by associated operations.
  \end{center}
\end{frame}
%%%%%%%%%%%%%%%%%%%%

%%%%%%%%%%%%%%%%%%%%
\begin{frame}{}
  \begin{center}
    \begin{prop}[Compactness of CJupiter (Informal)]
      {\large At a high level, CJupiter maintains only \red{one} $n$-ary ordered state space.}
    \end{prop}

    \vspace{0.20cm}
    % \resizebox{0.50\textwidth}{!}{% file: tikz/cjupiter-allinone-path.tex
% CSS lattice for jupiter-scheduling-podc16.tex

% default horizontal/vertical distance
\def\hdist{1.8}
\def\vdist{1.8}

\newcommand{\state}[2]{% #1: state label; #2: position
  \node (#1) [circle, inner sep = 0pt, minimum size = 10mm, text width = 10mm, align = center, draw, #2, font = \Large] {$#1$};
}

\tikzset{every lower node part/.style = {red}}
\newcommand{\statesplit}[3]{% #1: state upper label; #2: state lower label; #3: position
  \node (#1) [circle split, draw, minimum size = 6mm, text width = 10mm, align = center, #3, font = \Large]
  {
	$#1$
	\nodepart{lower}
	$#2$
  };
}

\newcommand{\transition}[4][]{% #2: start state; #3: end state; #4: transition label; #1: transition label position (optional)
  \draw[>=Stealth, ->] (#2) to node [rectangle, draw, above = 2pt, sloped, #1, font = \small] {#4} (#3);
}

\newcommand{\ins}[2]{$\textsc{Ins}(#1,#2)$}
\newcommand{\del}[2]{$\textsc{Del}(#1,#2)$}

\tikzset{node distance = \vdist and \hdist}
\tikzset{path/.style = {draw, rounded corners, ultra thick, #1}}

\begin{tikzpicture}
  \uncover<1->{
    \statesplit{0}{\epsilon}{}
    \statesplit{1}{x}{below = of 0}
    \transition{0}{1}{1: \ins{x}{0}}

    \statesplit{12}{\epsilon}{below left = of 1}
    \statesplit{13}{ax}{below right = of 1}
    % \statesplit{123}{a}{below = 2.1*\vdist of 1}
    \statesplit{123}{a}{below right = of 12}
    \transition{1}{12}{2: \del{x}{0}}
    \transition{1}{13}{3: \ins{a}{0}}
    \transition{12}{123}{3: \ins{a}{0}}
    \transition{13}{123}{2: \del{x}{1}}

    % \statesplit{14}{xb}{right = 1.5*\vdist of 13}
    % \transition{1}{14}{4: \ins{b}{1}}

    % \statesplit{124}{b}{right = 2.1*\hdist of 123}
    \statesplit{124}{b}{below right = of 13}
    \transition[near end]{12}{124}{4: \ins{b}{0}}
    
    % \statesplit{1234}{ba}{below = 2*\vdist of 13}
    \statesplit{1234}{ba}{below right = of 123}
    \transition{124}{1234}{3: \ins{a}{1}}
    \transition{123}{1234}{4: \ins{b}{0}}

    \statesplit{14}{xb}{above right = of 124}
    \transition{1}{14}{4: \ins{b}{1}}
    \transition{14}{124}{2: \del{x}{0}}
  }

  \uncover<2->{
    \path[path = {red}] ($(0.north) + (135:20pt)$) node[left] 
	  (sc1) {$S, C_1$} -- ++(-90:2.5*\vdist) -- ++(-135:2.5*\vdist) -- ++(-45:5.2*\vdist); % -- ++(-50:2.6*\vdist);
    \path[path = {blue}] ($(0.north) + (100:20pt)$) node[above = 0pt] 
	  (c2) {$C_2$} -- ++(-90:2.7*\vdist) -- ++(-45:2.5*\vdist) -- ++(-135:2.2*\vdist) -- ++(-45:2.4*\vdist);
    \path[path = {teal}] ($(0.north) + (60:15pt)$) node[right] 
	  (c3) {$C_3$} -- ++(-90:2.5*\vdist) -- ++(-18:5.5*\vdist) -- ++(-135:5.0*\vdist);
  }
\end{tikzpicture}}
    \fignocaption{width = 0.40\textwidth}{figs/cjupiter-allinone-path}

    \vspace{0.20cm}
    Each replica behavior corresponds to a \red{path} going through this state space.
  \end{center}
\end{frame}
%%%%%%%%%%%%%%%%%%%%

%%%%%%%%%%%%%%%%%%%%
\begin{frame}{}
  \begin{Theorem}[Equivalence of CJupiter and Jupiter]
    Under the same schedule, the behaviors of corresponding replicas in CJupiter and Jupiter are the same.
  \end{Theorem}

  \vspace{1.0cm}
  \centerline{\large From the perspectives of both the server and the clients.}
\end{frame}
%%%%%%%%%%%%%%%%%%%%
%%%%%%%%%%%%%%%%%%%%
\begin{frame}{}
  \begin{center}
    {\large We focus on a single $n$-ary ordered state space.}
  \end{center}

  \fignocaption{width = 0.45\textwidth}{figs/cjupiter-allinone-path}

  \begin{center}
    {\large We show the \red{pairwise state compatibility} property \blue{in three steps}.}
  \end{center}
\end{frame}
%%%%%%%%%%%%%%%%%%%%

%%%%%%%%%%%%%%%%%%%%
\begin{frame}{}
  \centerline{\circled{1} Take any two nodes/states $v_1$ and $v_2$.}

  \begin{lemma}[LCA (Lowest Common Ancestor)]
    Each pair of states in the $n$-ary ordered state space has a \red{unique} LCA.
  \end{lemma}

  \begin{columns}
    \column{0.60\textwidth}
      \fignocaption{width = 0.50\textwidth}{figs/lca}
      \column{0.30\textwidth}
	\[
	  v_0 = \text{LCA}(v_1, v_2)
	\]
  \end{columns}
\end{frame}
%%%%%%%%%%%%%%%%%%%%

%%%%%%%%%%%%%%%%%%%%
\begin{frame}{}
  \centerline{\circled{2} Consider the paths to $v_1$ and $v_2$ from their LCA $v_0$.}

  \begin{lemma}[Disjoint Paths]
    The set of operations $O_{v_0 \leadsto v_1}$ along $P_{v_0 \leadsto v_1}$ 
    is \red{disjoint} from the set of operations $O_{v_0 \leadsto v_2}$ along $P_{v_0 \leadsto v_2}$.
  \end{lemma}

  \begin{columns}
    \column{0.60\textwidth}
	\fignocaption{width = 0.50\textwidth}{figs/disjoint-paths}
      \column{0.30\textwidth}
	\[
	  v_0 = \text{LCA}(v_1, v_2)
	\]
  \end{columns}
\end{frame}
%%%%%%%%%%%%%%%%%%%%

%%%%%%%%%%%%%%%%%%%%
\begin{frame}{}
  \centerline{\circled{3} Consider the states in these two paths.}

  \begin{lemma}[Compatible Paths]
    Each pair of states consisting of one state $v$ in $P_{v_0 \leadsto v_1}$ and the other $v'$ in $P_{v_0 \leadsto v_2}$ are \red{compatible}.
  \end{lemma}

  \begin{columns}
    \column{0.60\textwidth}
	\fignocaption{width = 0.55\textwidth}{figs/compatible-paths}
      \column{0.40\textwidth}
	\[
	  v_0 = \text{LCA}(v_1, v_2)
	\]

	\vspace{0.50cm}
	\begin{center}
	  {In particular, \\ $v_1$ and $v_2$ are compatible.}
	\end{center}
  \end{columns}
\end{frame}
%%%%%%%%%%%%%%%%%%%%


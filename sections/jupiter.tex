% file: sections/jupiter.tex

\subsection{Jupiter}

%%%%%%%%%%%%%%%%%%%%
\begin{frame}{}
  \begin{center}
    \begin{mdframed}[frametitle = {\large Brief Announcement @ PODC'2018~\footnotemark}, frametitlerule = true, frametitlebackgroundcolor = brown!20,
      frametitleaboveskip = 8pt, frametitlebelowskip = 8pt, innertopmargin = 10pt]
      {\Large 实现复制列表的 \blue{Jupiter 协议}~\ncite{Nichols:UIST95}\footfullcite{Nichols:UIST95} 
      \red{满足} \blue{weak list specification}~\ncite{Attiya:PODC16}\footfullcite{Attiya:PODC16}.} \\[15pt]
    \end{mdframed}

    \footnotetext{\large \uncover<2->{\blue{藏在脚注里的}}\red{猜想} @ PODC'2016 \ncite{Attiya:PODC16}}
  \end{center}
\end{frame}
%%%%%%%%%%%%%%%%%%%%

%%%%%%%%%%%%%%%%%%%%
\begin{frame}{}
  \centerline{\Huge \teal{Replicated List}}
\end{frame}
%%%%%%%%%%%%%%%%%%%%

%%%%%%%%%%%%%%%%%%%%
\begin{frame}{}
  \centerline{\teal{\Large 基于\red{副本}的协同文本编辑系统}}

  \fignocaption{width = 0.55\textwidth}{figs/coeditor}

  \pause
  \vspace{-0.40cm}
  \begin{center}
    \begin{itemize}
      \centering
      \item 副本节点要\red{立即}响应本地用户操作 \\[4pt]
      \item 更新操作\red{异步}传播到其它副本节点
    \end{itemize}
  \end{center}
\end{frame}
%%%%%%%%%%%%%%%%%%%%

%%%%%%%%%%%%%%%%%%%%
\begin{frame}{}
  \centerline{\Large \red{复制列表对象}: 建模编辑系统的核心功能}
  \vspace{0.30cm}

  \begin{center}
    \begin{minipage}{0.70\textwidth}
      \begin{description}
	\setlength{\itemsep}{10pt}
	\item[$\textsc{Ins}(a, p):$] 在 $p$ 位置插入元素 $a$
	\item[$\textsc{Del}(p):$] 删除 $p$ 位置上的元素
	\item[$\textsc{Read}:$] 返回该列表
      \end{description}
    \end{minipage}
  \end{center}
\end{frame}
%%%%%%%%%%%%%%%%%%%%

%%%%%%%%%%%%%%%%%%%%
\begin{frame}{}
  \begin{cdef}[最终收敛性 (Eventual Convergence)~\ncite{Ellis:SIGMOD89}]
    当用户不再提交更新操作时, 每个副本节点上的列表是相同的。
  \end{cdef}

  \pause
  \vspace{0.50cm}

  \begin{cdef}[强最终一致性 (Strong Eventual Consistency)~\ncite{Shapiro:SSS11}]
    如果两个副本节点处理了同一组用户操作, 那么这两个副本节点上对列表是相同的。
  \end{cdef}

  \pause
  \vspace{0.60cm}
  \centerline{\red{\large 对系统的\red{中间状态}缺少足够的约束}}
\end{frame}
%%%%%%%%%%%%%%%%%%%%

%%%%%%%%%%%%%%%%%%%%
\begin{frame}{}
  \fignocaption{width = 0.80\textwidth, frame}{figs/podc16-attiya}

  \vspace{0.20cm}
  \begin{cdef}[Weak List Specification \wlspec{}~\ncite{Attiya:PODC16}]
    Informally, \wlspec{} requires the ordering between \red{elements that are not deleted} to be consistent across the system.
  \end{cdef}

  \vspace{0.60cm}
  \centerline{\teal{\large 定义在系统所有列表状态上的\red{全局}性质}}
\end{frame}
%%%%%%%%%%%%%%%%%%%%

%%%%%%%%%%%%%%%%%%%%
\begin{frame}{}
  \centerline{\blue{\Large 状态对兼容性 {\normalsize (Pairwise State Compatibility Property)}:}}

  \vspace{0.40cm}
  \begin{quote}
    For any \red{pair} of list states, there \red{cannot} be two elements $a$ and $b$ such that $a$ precedes $b$ in one state
    but $b$ precedes $a$ in the other.
  \end{quote}

  \vspace{0.30cm}
  \begin{columns}
    \column{0.50\textwidth}
      \fignocaption{width = 0.45\textwidth}{figs/ex-weak-list-spec}
      \vspace{-0.60cm}
      \fignocaption{width = 0.30\textwidth}{figs/red-cross}
    \pause
    \column{0.50\textwidth}
      \fignocaption{width = 0.55\textwidth}{figs/ex-strong-list-spec}
      \vspace{-0.60cm}
      \fignocaption{width = 0.20\textwidth}{figs/green-check}
  \end{columns}
\end{frame}
%%%%%%%%%%%%%%%%%%%%

%%%%%%%%%%%%%%%%%%%%
\begin{frame}{}
  \centerline{\large Jupiter 采用 \red{Client-Server} 架构~\ncite{Nichols:UIST95}}

  \begin{center}
    \begin{minipage}{0.40\textwidth}
      \input{tikz/jupiter-cs-tikz}
    \end{minipage}
  \end{center}

  \vspace{-0.60cm}
  \uncover<2->{
    \begin{center}
      服务器负责将所有操作序列化 \\[6pt]
      用户操作 $\xrightarrow[\text{\hl{立即返回}}]{\quad \text{提交} \quad}$ 客户 
      $\xrightarrow[\text{更新操作}]{\quad \text{FIFO} \quad}$ 服务器 $\xrightarrow[\text{\hl{更新操作}}]{\quad \text{FIFO} \quad}$ 其它客户 
    \end{center}
  }
\end{frame}
%%%%%%%%%%%%%%%%%%%%

%%%%%%%%%%%%%%%%%%%%
\begin{frame}{}
  \centerline{\large \red{操作转换} {\normalsize (Operational Transformation; OT)}~\ncite{Ellis:SIGMOD89} 技术}

  \begin{columns}
    \column{0.40\textwidth}
      \begin{center}
	\input{tikz/no-ot-tcs06-tikz}
      \end{center}
    \column{0.40\textwidth}
      \begin{center}
	\input{tikz/ot-tcs06-tikz}
      \end{center}
  \end{columns}
\end{frame}
%%%%%%%%%%%%%%%%%%%%

%%%%%%%%%%%%%%%%%%%%
\begin{frame}{}
  \fignocaption{width = 0.45\textwidth}{figs/ot}
  \begin{equation*}
    \resizebox{0.40\textwidth}{!}{$\sigma; o_1; o_2' \equiv \sigma; o_2; o_1'$}
  \end{equation*}
\end{frame}
%%%%%%%%%%%%%%%%%%%%

%%%%%%%%%%%%%%%%%%%%
\begin{frame}{}
  \centerline{\large 针对列表的操作转换函数~\ncite{Ellis:SIGMOD89}}

  \resizebox{\textwidth}{!}{
    \begin{minipage}{\textwidth}
      \input{sections/list-ot}
    \end{minipage}
  }
\end{frame}
%%%%%%%%%%%%%%%%%%%%

%%%%%%%%%%%%%%%%%%%%
\begin{frame}{}
  \fignocaption{width = 0.75\textwidth}{figs/ot-diverge-two}
\end{frame}
%%%%%%%%%%%%%%%%%%%%

%%%%%%%%%%%%%%%%%%%%
\begin{frame}{}
  \begin{center}
    {\large 利用数据结构 \red{$2D$ 状态空间}~\ncite{Xu:CSCW14} \\
    控制何时以及如何执行``操作转换''}
  \end{center}

  \fignocaption{width = 0.60\textwidth}{figs/ot-diverge-two}
  % \fignocaption{width = 0.32\textwidth}{figs/2d-statespace}

  \begin{center}
    $2D$: {\textsc{Local}} \emph{vs.} \textsc{Global}
  \end{center}
\end{frame}
%%%%%%%%%%%%%%%%%%%%

%%%%%%%%%%%%%%%%%%%%
\begin{frame}{}
  \centerline{\large 每个 \red{Client} 维护一个 $2D$ 状态空间}

  % \fignocaption{width = 0.50\textwidth}{figs/jupiter-illustration-client3-notations}
  \fignocaption{width = 0.70\textwidth}{figs/jupiter-illustration}

  \centerline{\large \red{Server} 维护 $n \; (=3)$ 个 $2D$ 状态空间, 与 $n$ 个 Clients 对应}
\end{frame}
%%%%%%%%%%%%%%%%%%%%

%%%%%%%%%%%%%%%%%%%%
\begin{frame}{}
  \begin{center}
    {\large \red{Global property} on all replica states specified by \blue{\wlspec{}}}

    \vspace{0.20cm}
    \fignocaption{width = 0.30\textwidth}{figs/mismatch}
    \vspace{0.20cm}

    {\large \red{Local view} each replica maintains in \blue{Jupiter}}
  \end{center}
\end{frame}
%%%%%%%%%%%%%%%%%%%%

%%%%%%%%%%%%%%%%%%%%
\begin{frame}{}
  \begin{center}
    {\large CJupiter maintains an \red{$n$-ary ordered state space} for each replica.}
  \end{center}

  \fignocaption{width = 0.40\textwidth}{figs/cjupiter-allinone}

  \begin{center} 
    There can be \red{more than two edges} coming from the same node. \\[5pt]
    Edges from the same node are \red{totally ordered} by associated operations.
  \end{center}
\end{frame}
%%%%%%%%%%%%%%%%%%%%

%%%%%%%%%%%%%%%%%%%%
\begin{frame}{}
  \begin{center}
    \begin{prop}[Compactness of CJupiter (Informal)]
      {\large At a high level, CJupiter maintains only \red{one} $n$-ary ordered state space.}
    \end{prop}

    \vspace{0.20cm}
    % \resizebox{0.50\textwidth}{!}{\input{tikz/cjupiter-allinone-path}}
    \fignocaption{width = 0.40\textwidth}{figs/cjupiter-allinone-path}

    \vspace{0.20cm}
    Each replica behavior corresponds to a \red{path} going through this state space.
  \end{center}
\end{frame}
%%%%%%%%%%%%%%%%%%%%

%%%%%%%%%%%%%%%%%%%%
\begin{frame}{}
  \begin{Theorem}[Equivalence of CJupiter and Jupiter]
    Under the same schedule, the behaviors of corresponding replicas in CJupiter and Jupiter are the same.
  \end{Theorem}

  \vspace{1.0cm}
  \centerline{\large From the perspectives of both the server and the clients.}
\end{frame}
%%%%%%%%%%%%%%%%%%%%
%%%%%%%%%%%%%%%%%%%%
\begin{frame}{}
  \begin{center}
    {\large We focus on a single $n$-ary ordered state space.}
  \end{center}

  \fignocaption{width = 0.45\textwidth}{figs/cjupiter-allinone-path}

  \begin{center}
    {\large We show the \red{pairwise state compatibility} property \blue{in three steps}.}
  \end{center}
\end{frame}
%%%%%%%%%%%%%%%%%%%%

%%%%%%%%%%%%%%%%%%%%
\begin{frame}{}
  \centerline{\circled{1} Take any two nodes/states $v_1$ and $v_2$.}

  \begin{lemma}[LCA (Lowest Common Ancestor)]
    Each pair of states in the $n$-ary ordered state space has a \red{unique} LCA.
  \end{lemma}

  \begin{columns}
    \column{0.60\textwidth}
      \fignocaption{width = 0.50\textwidth}{figs/lca}
      \column{0.30\textwidth}
	\[
	  v_0 = \text{LCA}(v_1, v_2)
	\]
  \end{columns}
\end{frame}
%%%%%%%%%%%%%%%%%%%%

%%%%%%%%%%%%%%%%%%%%
\begin{frame}{}
  \centerline{\circled{2} Consider the paths to $v_1$ and $v_2$ from their LCA $v_0$.}

  \begin{lemma}[Disjoint Paths]
    The set of operations $O_{v_0 \leadsto v_1}$ along $P_{v_0 \leadsto v_1}$ 
    is \red{disjoint} from the set of operations $O_{v_0 \leadsto v_2}$ along $P_{v_0 \leadsto v_2}$.
  \end{lemma}

  \begin{columns}
    \column{0.60\textwidth}
	\fignocaption{width = 0.50\textwidth}{figs/disjoint-paths}
      \column{0.30\textwidth}
	\[
	  v_0 = \text{LCA}(v_1, v_2)
	\]
  \end{columns}
\end{frame}
%%%%%%%%%%%%%%%%%%%%

%%%%%%%%%%%%%%%%%%%%
\begin{frame}{}
  \centerline{\circled{3} Consider the states in these two paths.}

  \begin{lemma}[Compatible Paths]
    Each pair of states consisting of one state $v$ in $P_{v_0 \leadsto v_1}$ and the other $v'$ in $P_{v_0 \leadsto v_2}$ are \red{compatible}.
  \end{lemma}

  \begin{columns}
    \column{0.60\textwidth}
	\fignocaption{width = 0.55\textwidth}{figs/compatible-paths}
      \column{0.40\textwidth}
	\[
	  v_0 = \text{LCA}(v_1, v_2)
	\]

	\vspace{0.50cm}
	\begin{center}
	  {In particular, \\ $v_1$ and $v_2$ are compatible.}
	\end{center}
  \end{columns}
\end{frame}
%%%%%%%%%%%%%%%%%%%%


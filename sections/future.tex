% file: sections/future.tex

\section{未来工作}

%%%%%%%%%%%%%%%
\begin{frame}{}
  \fignocaption{width = 0.618\textwidth}{figs/one-more-thing}
\end{frame}
%%%%%%%%%%%%%%%

%%%%%%%%%%%%%%%
\begin{frame}{}
  \fignocaption{width = 0.65\textwidth}{figs/rdt-research-spec}
\end{frame}
%%%%%%%%%%%%%%%

%%%%%%%%%%%%%%%
\begin{frame}{}
  \begin{center}
    \uncover<2->{\hl{\large 规约: 数据一致性模型 (Consistency Model)}}
  \end{center}

  \begin{columns}
    \column{0.50\textwidth}
      \begin{center}
	多处理器系统中的并发数据类型
      \end{center}
      \vspace{0.60cm}
      \fignocaption{width = 0.50\textwidth}{figs/cdt}
    \column{0.50\textwidth}
      \begin{center}
	分布式系统中的复制数据类型
      \end{center}
      \fignocaption{width = 0.60\textwidth}{figs/rdt}
  \end{columns}

  \vspace{0.20cm}
  \uncover<3->{
    \begin{center}
      \red{\LARGE PL + DC \uncover<4->{+ FM}}	\ncite{Burckhardt:POPL14}
    \end{center}
  }
\end{frame}
%%%%%%%%%%%%%%%

%%%%%%%%%%%%%%%
\begin{frame}{}
  \begin{center}
    ($50$种) \hl{一致性模型}关系图 \ncite{Viotti:CSUR16} \ncite{Burckhardt:Book14}
  \end{center}

  \fignocaption{width = 0.95\textwidth}{figs/non-transactional-consistency-models}

  \centerline{\red{\large 建立统一的形式化框架}}
\end{frame}
%%%%%%%%%%%%%%%

% file: sections/background.tex

\section{研究背景}

%%%%%%%%%%%%%%%
\begin{frame}{}
  \begin{center}
    {\large Abstract Data Types} (\textsc{ADT}) \ncite{Liskov:VHLL74} \\[8pt]

    (单线程; 顺序语义)
  \end{center}

  \begin{columns}
    \column{0.40\textwidth}
      \fignocaption{width = 0.60\textwidth}{figs/alg-ds-wirth}
    \pause
    \column{0.20\textwidth}
      \fignocaption{width = 0.50\textwidth}{figs/adt}
    \pause
    \column{0.40\textwidth}
      \fignocaption{frame, width = 0.65\textwidth}{figs/simple-not-easy}
  \end{columns}
\end{frame}
%%%%%%%%%%%%%%%

%%%%%%%%%%%%%%%
\begin{frame}{}
  \begin{center}
    {\large Concurrent Data Types} \ncite{Herlihy:TOPLAS90} \\[8pt]

    (多线程; 并发语义)
  \end{center}

  \begin{columns}
    \column{0.40\textwidth}
      \fignocaption{width = 0.70\textwidth}{figs/taomp-herlihy}
    \pause
    \column{0.20\textwidth}
      \fignocaption{width = 0.90\textwidth}{figs/cdt}
    \pause
    \column{0.40\textwidth}
      \fignocaption{width = 0.85\textwidth}{figs/crack-computer}
  \end{columns}
\end{frame}
%%%%%%%%%%%%%%%

%%%%%%%%%%%%%%%
\begin{frame}{}
  \begin{center}
    \hl{\Large Replicated Data Types} {(\textsc{RDT}; \textcolor{red}{$\approx$ 2010 年})} \ncite{Burckhardt:POPL14} \\[8pt]

    (多副本; 复制语义)
  \end{center}

  \pause
  \begin{columns}
    \column{0.50\textwidth}
      \fignocaption{width = 0.60\textwidth}{figs/rdt}
    \pause
    \column{0.50\textwidth}
      \fignocaption{width = 0.80\textwidth}{figs/complicated}
  \end{columns}
\end{frame}
%%%%%%%%%%%%%%%

%%%%%%%%%%%%%%%
\begin{frame}{}
  \begin{columns}
    \column{0.50\textwidth}
      \fignocaption{width = 0.60\textwidth}{figs/keep-calm-why-bother}
    \pause
    \column{0.50\textwidth}
      \centerline{\hl{\red{\Huge 新平台}}}
  \end{columns}
\end{frame}
%%%%%%%%%%%%%%%

%%%%%%%%%%%%%%%
\begin{frame}{}
  \centerline{\Large 大规模分布式系统}

  \fignocaption{width = 0.45\textwidth}{figs/sina-weibo-world-map.pdf}
  \vspace{0.50cm}

  \begin{columns}
    \column{0.40\textwidth}
    新浪微博社交应用~\footnotemark:
    \begin{itemize}
      \item 日均用户近一亿名
      \item 日均消息近一亿条
    \end{itemize}
    \pause
    \column{0.50\textwidth}
    特性需求: 
    \begin{itemize}
      \item 低延迟, 高可用性 (4个9~\footnotemark)
      \item 高容错性, 高可扩展性
    \end{itemize}
  \end{columns}
  
  \footnotetext[1]{2015第三季度; 数据来自 \href{https://www.chinainternetwatch.com/15740/weibo-q3-2015/}{China Internet Watch}.}
  % See http://tex.stackexchange.com/a/340079/23098
  \alt<1>{\let\thefootnote\relax\footnotetext{~}}{\footnotetext[2]{数据来自 \href{http://www.infoq.com/cn/articles/weibo-platform-availability-9999}{InfoQ}.}}
\end{frame}
%%%%%%%%%%%%%%%
\begin{frame}{}
  \graphicspath{{tikz/}}
  \begin{figure}[h!]
    \centering
    \begin{adjustbox}{max totalsize = {0.50\textwidth}{1.00\textheight}, center}
      \input{tikz/distributed-data-overlay-for-background}
    \end{adjustbox}
  \end{figure}

  \vspace{0.20cm}
  \begin{center}
    \begin{minipage}{0.65\textwidth}
      \red{\large 分布数据 \term{distributed data}:}

      \vspace{0.20cm}
      \begin{enumerate}
	\item<2-> 分区 \term{partition}: 水平扩展
	\item<3-> \hl{副本 \term{replication}}: 就近访问, 容灾备份
      \end{enumerate}
    \end{minipage}
  \end{center}
\end{frame}
%%%%%%%%%%%%%%%

%%%%%%%%%%%%%%%
\begin{frame}{}
  \begin{center}
    \begin{minipage}{0.50\textwidth}
      {\large 复制数据类型} \ncite{Shapiro:TR11} 

      \vspace{0.20cm}
      \begin{itemize}
	\setlength{\itemsep}{4pt}
	\item \hl{Read/Write Register}
	\item Counter
	\item Set
	\item \hl{List}
	\item HashMap
	\item Disjoint Set
	\item Graph
	\item $\dots$
      \end{itemize}
    \end{minipage}
  \end{center}
\end{frame}
%%%%%%%%%%%%%%%

%%%%%%%%%%%%%%%
\begin{frame}{}
  \begin{columns}
    \column{0.40\textwidth}
      \fignocaption{width = 0.98\textwidth}{figs/what-is-new}
    \column{0.60\textwidth}
      \centerline{\hl{\red{\Huge 新问题, 新挑战}}}
  \end{columns}

  \pause
  \vspace{0.30cm}
  \fignocaption{frame, width = 0.80\textwidth}{figs/rdt-popl}
  \centerline{\ncite{Burckhardt:POPL14}}
\end{frame}
%%%%%%%%%%%%%%%

%%%%%%%%%%%%%%%
\begin{frame}{}
  \fignocaption{width = 0.65\textwidth}{figs/rdt-research-two-work}
\end{frame}
%%%%%%%%%%%%%%%

% %%%%%%%%%%%%%%%
% \begin{frame}{分布共享数据服务典型应用 (II)}
%   \fig{width = 0.75\textwidth}{figs/file-share.pdf}
%   {个人多设备文件共享 {(\textcolor{blue}{\scriptsize [基于云] C/S 结构 [左]} \& 
%   \textcolor{red}{\scriptsize P2P 结构 [右]}).}}
% 
%   \begin{description}
%     \item[功能需求:] 文件副本 \citeinbeamer{Strauss}{MIT Thesis}{10}
%     \item[网络断连:] 备份容灾; 离线可用
%   \end{description}
% \end{frame}
%%%%%%%%%%%%%%%
% \begin{frame}{分布副本数据的典型应用 (三)}
%   \fig{width = 0.80\textwidth}{figs/coordination.pdf}
%   {分布式协同应用(上)及服务(下).}
% 
%   分布式协同服务需求 \citeinbeamer{Burrows}{OSDI}{06} vs. ``副本技术'':
%   \begin{description}
%     \item[高性能:] 低延迟就近``读''副本 \citeinbeamer{Yahoo!}{USENIXATC}{10}
%     \item[高可靠性:] 避免单点协同故障
%   \end{description}
% \end{frame}
%%%%%%%%%%%%%%%

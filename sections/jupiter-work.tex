% file: sections/jupiter-work.tex

%%%%%%%%%%%%%%%%%%%%
\begin{frame}{}
  \begin{center}
    \begin{mdframed}[frametitle = {\large Brief Announcement @ PODC'2018~\footnotemark}, frametitlerule = true, frametitlebackgroundcolor = brown!20,
      frametitleaboveskip = 8pt, frametitlebelowskip = 8pt, innertopmargin = 10pt]
      {\large 实现复制列表的 \blue{Jupiter 协议}~\ncite{Nichols:UIST95}\footfullcite{Nichols:UIST95} \red{满足} \\
      \blue{weak list specification}~\ncite{Attiya:PODC16}\footfullcite{Attiya:PODC16}.} \\[15pt]
    \end{mdframed}

    \footnotetext{\normalsize \uncover<2->{藏在脚注里的}\red{猜想}@PODC'2016~\ncite{Attiya:PODC16}}
  \end{center}
\end{frame}
%%%%%%%%%%%%%%%%%%%%

%%%%%%%%%%%%%%%%%%%%
\begin{frame}{}
  \centerline{\Huge \teal{Weak List Specification}}
\end{frame}
%%%%%%%%%%%%%%%%%%%%

%%%%%%%%%%%%%%%%%%%%
\begin{frame}{}
  \centerline{\teal{\Large 基于\red{副本}的协同文本编辑系统}}

  % \vspace{-0.20cm}
  \fignocaption{width = 0.55\textwidth}{figs/coeditor}

  % \pause
  % \vspace{-0.40cm}
  % \begin{center}
  %   \begin{itemize}
  %     \centering
  %     \item replica要\red{立即}响应本地用户操作 \\[4pt]
  %     \item 更新操作\red{异步}传播到其它replica
  %   \end{itemize}
  % \end{center}
\end{frame}
%%%%%%%%%%%%%%%%%%%%

%%%%%%%%%%%%%%%%%%%%
\begin{frame}{}
  \centerline{\Large \red{复制列表对象}: 建模编辑系统的核心功能}
  \vspace{0.30cm}

  \begin{center}
    \begin{minipage}{0.70\textwidth}
      \begin{description}
	\setlength{\itemsep}{10pt}
	\item[$\textsc{Ins}(a, p):$] 在 $p$ 位置插入元素 $a$
	\item[$\textsc{Del}(p):$] 删除 $p$ 位置上的元素
	\item[$\textsc{Read}:$] 返回该列表
      \end{description}
    \end{minipage}
  \end{center}
\end{frame}
%%%%%%%%%%%%%%%%%%%%

%%%%%%%%%%%%%%%%%%%%
\begin{frame}{}
  \fignocaption{width = 0.80\textwidth, frame}{figs/podc16-attiya}

  \vspace{0.20cm}
  \begin{cdef}[Weak List Specification \wlspec{}~\ncite{Attiya:PODC16}]
    Informally, \wlspec{} requires the ordering between \red{elements that are not deleted} to be consistent across the system.
  \end{cdef}

  \vspace{0.60cm}
  \centerline{\teal{\large 定义在系统所有列表状态上的\red{全局}性质}}
\end{frame}
%%%%%%%%%%%%%%%%%%%%

%%%%%%%%%%%%%%%%%%%%
\begin{frame}{}
  \begin{cdef}[\hl{状态对兼容性} (Pairwise State Compatibility Property)]
    任给两个列表状态 $\sigma_0$、$\sigma_1$, 若它们含有两个共同元素 $a$、$b$, \\
    则 $a$、$b$ 在 $\sigma_0$ 与 $\sigma_1$ 中的相对顺序保持一致。
  \end{cdef}

  \vspace{0.30cm}
  \begin{columns}
    \column{0.50\textwidth}
      \fignocaption{width = 0.50\textwidth}{figs/ex-weak-list-spec}
      \vspace{-0.60cm}
      \fignocaption{width = 0.30\textwidth}{figs/red-cross}
    \pause
    \column{0.50\textwidth}
      \fignocaption{width = 0.65\textwidth}{figs/ex-strong-list-spec}
      \vspace{-0.60cm}
      \fignocaption{width = 0.20\textwidth}{figs/green-check}
  \end{columns}
\end{frame}
%%%%%%%%%%%%%%%%%%%%

%%%%%%%%%%%%%%%%%%%%
\begin{frame}{}
  \centerline{\Huge \teal{Jupiter}}
\end{frame}
%%%%%%%%%%%%%%%%%%%%

%%%%%%%%%%%%%%%%%%%%
\begin{frame}{}
  \centerline{\large $\blue{(n+1)}\; \text{replicas} \triangleq \blue{(n)}\; \text{\red{Client}} + \blue{(1)}\; \text{\red{Server}}$~\ncite{Nichols:UIST95}}

  \begin{center}
    \resizebox{0.50\textwidth}{!}{% file: tikz/jupiter-cs-tikz.tex

\def\s{s}  % server
\def\b{bot} % literal string
\tikzset{seq/.style = {draw, circle, outer sep = 5pt, inner sep = 2pt, scale = #1, left}}

% send: the sender sends operation to the receiver
\newcommand{\send}[5]{% #1: sender; #2: receiver; #3: sender pos; #4: receiver pos; #5: seq. number;
  \draw[->]  ($(#1)!#3!(#1\b)$) node[seq = 0.40] {#5} to ($(#2)!#4!(#2\b)$) node[seq = 0.60] {$#5$};
}

% csend: client sends operation to the server
\newcommand{\csend}[6]{% #1: client; #2: client pos; #3: server pos; #4: seq. number; #5: client state; #6: server state
  \draw[->]  ($(#1)!#2!(#1\b)$) node[seq = 0.60, draw = none] (#4) {} to ($(\s)!#3!(\s\b)$) {};
}

% ssend: the server sends operation to client 
\newcommand{\ssend}[5]{% #1: client; #2: server pos; #3: client pos; #4: seq. number; #5: client state
  \draw[->, dashed]  ($(\s)!#2!(\s\b)$) to ($(#1)!#3!(#1\b)$) node[solid, seq = 0.60] {$#4$};
}

\newcommand{\ins}[2]{$\textcolor{blue}{\textsc{Ins}(#1,#2)}$}
\newcommand{\del}[2]{$\textcolor{blue}{\textsc{Del}(#2)}$}  % del(a,p): ignoring the element 'a'

\begin{tikzpicture}[
    timeline/.style = {very thick}, >=Stealth, 
    op/.style = {font = \small, above left = -0.20cm and -0.40cm of #1, sloped},
    replica/.style = {align = center}]
  \uncover<1->{
    \node[replica] (\s) {\textcolor{brown}{$S$}};
    \node[replica, right = of s] (c1) {$\textcolor{brown}{C_1}$};
    \node[replica, right = of c1] (c2) {$\textcolor{brown}{C_2}$};
    \node[replica, right = of c2] (c3) {$\textcolor{brown}{C_3}$};

    \foreach \r/\rbot in {s/sbot, c1/c1bot, c2/c2bot, c3/c3bot} {
	  \node[below = 5.0cm of \r] (\rbot) {};
	  \draw[timeline] (\r) to (\rbot);
    }
    \draw[timeline, ultra thick, red] (s) to (sbot);
  }

  \uncover<2->{
    % Clients send messages (ordered by positions at server) to the server
    \csend{c1}{0.15}{0.20}{1}{x}{x}
    \csend{c1}{0.35}{0.40}{2}{}{}
    \csend{c2}{0.45}{0.55}{3}{ax}{a}
    \csend{c3}{0.50}{0.75}{4}{xb}{}
  }

  \uncover<2->{
    % Operations are totally ordered at the server.
    \foreach \num/\pos in {1/0.20, 2/0.40, 3/0.55, 4/0.75} {
      \node [seq = 0.80, fill = red!40] at ($(\s)!\pos!(\s\b)$) {$\num$};
    }
  }

  % \node (o1) [op = {1}, rotate = 7] {};
  % \node (o2) [op = {2}, rotate = 7] {};
  % \node (o3) [op = {3}, rotate = 8] {};
  % \node (o4) [op = {4}, rotate = 15] {};

  \uncover<2->{
    % The server sends messages (ordered by positions at server) to clients.
    \ssend{c2}{0.40}{0.65}{2}{a}
    \ssend{c3}{0.40}{0.70}{2}{b}

    \ssend{c2}{0.20}{0.30}{1}{x}
    \ssend{c3}{0.20}{0.25}{1}{x}


    \ssend{c1}{0.55}{0.60}{3}{a}
    \ssend{c3}{0.55}{0.90}{3}{}

    \ssend{c1}{0.75}{0.90}{4}{}
    \ssend{c2}{0.75}{0.90}{4}{}
  }
\end{tikzpicture}
}
  \end{center}

  \vspace{-0.60cm}
  \uncover<2->{
    \begin{center}
      Server 负责将所有操作序列化 \\[6pt]
      用户操作 $\xrightarrow[\text{\hl{立即返回}}]{\quad \text{提交} \quad}$ Client 
      $\xrightarrow[\text{更新操作}]{\quad \text{FIFO} \quad}$ Server $\xrightarrow[\text{更新操作}]{\quad \text{FIFO} \quad}$ 其它Clients 
    \end{center}
  }
\end{frame}
%%%%%%%%%%%%%%%%%%%%

%%%%%%%%%%%%%%%%%%%%
\begin{frame}{}
  \centerline{\hl{\red{\large 操作转换}} (Operational Transformation; OT)~\ncite{Ellis:SIGMOD89}}

  \begin{columns}
    \column{0.40\textwidth}
      \begin{center}
	% file: tikz/no-ot-tcs06.tex

\newcommand{\ins}[2]{\textsc{Ins}(#1,#2)}
\newcommand{\del}[2]{\textsc{Del}(#2)}  % del(a, p): ignoring the deleted element ``a''

\begin{tikzpicture}[
	timeline/.style = {thick, dashed}, 
	>=Stealth, 
	send/.style = {>=Stealth, ->},
	list/.style = {rectangle, draw, inner sep = 5pt, outer sep = 2pt, fill = #1, font = \large},
	op/.style = {font = \footnotesize, #1}
  ]
  \uncover<2->{
    \node[list = blue!20, label = {above:{$R_1$}}] (r1) {efecte}; 
    \node[list = blue!20, right = 1.20cm of r1, label = {above:{$R_2$}}] (r2) {efecte};

    \foreach \r/\rbot in {r1/r1bot, r2/r2bot} {
      \node[below = 4.0cm of \r] (\rbot) {};
    }

    \draw[timeline] ($(r1.south)+(0,-5pt)$) -- ($(r1bot.south)+(0,-15pt)$);
    \draw[timeline] ($(r2.south)+(0,-5pt)$) -- ($(r2bot.south)+(0,-15pt)$);
  }

  \uncover<3->{
    \node (ins) [op = purple] at ($(r1)!0.25!(r1bot)$) {$o_1 = \ins{f}{1}$};
    \node (del) [op = purple] at ($(r2)!0.25!(r2bot)$) {$o_2 = \del{e}{5}$};

    \node (r11) [list = teal!20] at ($(r1)!0.50!(r1bot)$) {effecte};
    \node (r21) [list = teal!20] at ($(r2)!0.50!(r2bot)$) {efect};
  }

  \uncover<4->{
    \node (ins') [op] at ($(r2)!0.75!(r2bot)$) {$o_1' = \ins{f}{1}$};
    \node (del') [op] at ($(r1)!0.75!(r1bot)$) {$o_2' = \del{e}{5}$};

    \draw[send] (ins) -- (ins');
    \draw[send] (del) -- (del');

    \node (r12) [list = red!20] at ($(r1)!1.00!(r1bot)$) {effece};
    \node (r22) [list = red!20] at ($(r2)!1.00!(r2bot)$) {effect};
  }
\end{tikzpicture}
      \end{center}
    \column{0.40\textwidth}
      \begin{center}
	% file: tikz/ot-tcs06.tex

\newcommand{\ins}[2]{\textsc{Ins}(#1,#2)}
\newcommand{\del}[2]{\textsc{Del}(#2)}  % del(a, p): ignoring the deleted element ``a''

\begin{tikzpicture}[
	timeline/.style = {thick, dashed}, 
	>=Stealth, 
	send/.style = {>=Stealth, ->},
	list/.style = {rectangle, draw, inner sep = 5pt, outer sep = 2pt, fill = #1, font = \large},
	op/.style = {font = \footnotesize, #1}
  ]

  \uncover<5->{
    \node[list = blue!20, label = {above:{$R_1$}}] (r1) {efecte}; 
    \node[list = blue!20, right = 1.20cm of r1, label = {above:{$R_2$}}] (r2) {efecte};

    \foreach \r/\rbot in {r1/r1bot, r2/r2bot} {
      \node[below = 4.0cm of \r] (\rbot) {};
    }

    \draw[timeline] ($(r1.south)+(0,-5pt)$) -- ($(r1bot.south)+(0,-15pt)$);
    \draw[timeline] ($(r2.south)+(0,-5pt)$) -- ($(r2bot.south)+(0,-15pt)$);

    \node (ins) [op = purple] at ($(r1)!0.25!(r1bot)$) {$o_1 = \ins{f}{1}$};
    \node (del) [op = purple] at ($(r2)!0.25!(r2bot)$) {$o_2 = \del{e}{5}$};

    \node (r11) [list = teal!20] at ($(r1)!0.50!(r1bot)$) {effecte};
    \node (r21) [list = teal!20] at ($(r2)!0.50!(r2bot)$) {efect};
  }

  \uncover<6->{
    \node (ins') [op] at ($(r2)!0.75!(r2bot)$) {$o_1' = \ins{f}{1}$};
    \node (del') [op, red, ellipse, draw] at ($(r1)!0.75!(r1bot)$) {$o_2' = \del{e}{6}$};

    \draw[send] (ins) -- (ins');
    \draw[send] (del) -- (del');

    \node (r12) [list = green!20] at ($(r1)!1.00!(r1bot)$) {effect};
    \node (r22) [list = green!20] at ($(r2)!1.00!(r2bot)$) {effect};
  }
\end{tikzpicture}

      \end{center}
  \end{columns}
\end{frame}
%%%%%%%%%%%%%%%%%%%%

%%%%%%%%%%%%%%%%%%%%
\begin{frame}{}
  \fignocaption{width = 0.40\textwidth}{figs/ot}

  \begin{equation*}
    \text{\hl{交换律}}\;\; \resizebox{0.35\textwidth}{!}{$\sigma; o_1; o_2' \equiv \sigma; o_2; o_1'$}
  \end{equation*}

  \centerline{\ncite{Ellis:SIGMOD89}}
\end{frame}
%%%%%%%%%%%%%%%%%%%%

%%%%%%%%%%%%%%%%%%%%
\begin{frame}{}
  \begin{center}
    \resizebox{0.75\textwidth}{!}{% file: ot-diverge-two.tex

\documentclass[tikz]{standalone}

\usetikzlibrary{shapes, positioning, arrows.meta, calc, intersections, backgrounds, fit}

% default horizontal/vertical distance
\def\hdist{2.5}
\def\vdist{2.5}
\tikzset{node distance = \vdist and \hdist}

\newcommand{\state}[3]{% #1: state name; #2: position; #3: state label
  \node (#1) [circle, inner sep = 0pt, minimum size = 8mm, text width = 12mm, align = center, draw, #2, font = \Large] {#3};
}

\newcommand{\trans}[5]{% #1: start state; #2: end state; #3: transition label; #4: transition label position; #5: style
  \draw[>=Stealth, ->,  thick, #5] (#1) to node [rectangle, draw, above = 5pt, sloped, fill = teal!20, #4, scale = 1.2] {#3} (#2);
}

\begin{document}
\begin{tikzpicture}
  \state{0}{fill = blue!20}{$0123$}
  \state{l}{below left = of 0.center}{$123$}
  \state{ll}{below left = of l.center}{$13$}
  \state{llr}{below right = of ll.center}{$1$}
  \state{r}{below right = of 0.center}{$012$}
  \state{lr}{below left = of r.center}{$12$}

  \trans{0.south west}{l.north east}{$\textsc{Del(0)}$}{}{}
  \trans{l.south west}{ll.north east}{$\textsc{Del(1)}$}{}{}

  \trans{0.south east}{r.north west}{$\textsc{Del(3)}$}{fill = red!20}{}

  \trans{l.south east}{lr.north west}{$\textsc{Del}(2)$}{below = 5pt, fill = red!20}{dashed}
  \trans{r.south west}{lr.north east}{$\textsc{Del}(0)$}{below = 5pt}{}

  \trans{ll.south east}{llr.north west}{$\textsc{Del}(1)$}{below = 5pt, fill = red!20}{dashed}
  \trans{lr.south west}{llr.north east}{$\textsc{Del}(1)$}{below = 5pt}{}

  \state{rr}{below right = of r.center}{}
  \trans{r.south east}{rr.north west}{$o$}{scale = 1.5, fill = red!20}{}
\end{tikzpicture}
\end{document}
}
  \end{center}
\end{frame}
%%%%%%%%%%%%%%%%%%%%

%%%%%%%%%%%%%%%%%%%%
\begin{frame}{}
  \begin{center}
    {\large 利用数据结构 \hl{$2D$ 状态空间}~\ncite{Xu:CSCW14} \\
    控制何时以及如何执行``操作转换''}
  \end{center}

  \fignocaption{width = 0.60\textwidth}{figs/ot-diverge-two}

  \begin{center}
    $2D$: {\textsc{Local}} \emph{vs.} \textsc{Global}
  \end{center}
\end{frame}
%%%%%%%%%%%%%%%%%%%%

%%%%%%%%%%%%%%%%%%%%
\begin{frame}{}
  \centerline{\large 每个 \red{Client} 维护一个 $2D$ 状态空间}

  \fignocaption{width = 0.70\textwidth}{figs/jupiter-illustration}

  \centerline{\large \red{Server} 维护 $n$ 个 $2D$ 状态空间, 与 $n$ 个 Clients 对应}
\end{frame}
%%%%%%%%%%%%%%%%%%%%

%%%%%%%%%%%%%%%%%%%%
\begin{frame}{}
  \begin{center}
    \hl{\Huge Mismatch!}

    \vspace{1.00cm}
    {\large \blue{\wlspec{}} 所规定的\red{全局性质}}

    \vspace{0.20cm}
    \fignocaption{width = 0.40\textwidth}{figs/mismatch}
    \vspace{0.20cm}

    {\large \blue{Jupiter} 协议中, 每个replica所维护的\red{局部视图}}
  \end{center}
\end{frame}
%%%%%%%%%%%%%%%%%%%%

%%%%%%%%%%%%%%%%%%%%
\begin{frame}{}
  \centerline{\LARGE \teal{CJupiter (Compact Jupiter)}}

  \pause
  \vspace{1.0cm}
  \begin{ctheorem}[等价性]
    在相同的操作调度下, \emph{CJupiter} 与 \emph{Jupiter} 中的对应replica的行为 {\small (状态序列)} 是相同的。
  \end{ctheorem}
\end{frame}
%%%%%%%%%%%%%%%%%%%%

%%%%%%%%%%%%%%%%%%%%
\begin{frame}{}
  \begin{center}
    {\large CJupiter 为每个replica维护一个 \hl{\red{$n$-ary} \blue{有序}状态空间}}
  \end{center}

  \fignocaption{width = 0.50\textwidth}{figs/cjupiter-allinone}
\end{frame}
%%%%%%%%%%%%%%%%%%%%

%%%%%%%%%%%%%%%%%%%%
\begin{frame}{}
  \begin{center}
    \begin{prop}[Compactness of CJupiter]
      {\large \emph{CJupiter} 所维护的 $(n+1)$ 个$n$-ary有序状态空间是相同的。}
    \end{prop}

    \resizebox{0.50\textwidth}{!}{% file: tikz/cjupiter-allinone-path.tex
% CSS lattice for jupiter-scheduling-podc16.tex

% default horizontal/vertical distance
\def\hdist{1.8}
\def\vdist{1.8}

\newcommand{\state}[2]{% #1: state label; #2: position
  \node (#1) [circle, inner sep = 0pt, minimum size = 10mm, text width = 10mm, align = center, draw, #2, font = \Large] {$#1$};
}

\tikzset{every lower node part/.style = {red}}
\newcommand{\statesplit}[3]{% #1: state upper label; #2: state lower label; #3: position
  \node (#1) [circle split, draw, minimum size = 6mm, text width = 10mm, align = center, #3, font = \Large]
  {
	$#1$
	\nodepart{lower}
	$#2$
  };
}

\newcommand{\transition}[4][]{% #2: start state; #3: end state; #4: transition label; #1: transition label position (optional)
  \draw[>=Stealth, ->] (#2) to node [rectangle, draw, above = 2pt, sloped, #1, font = \small] {#4} (#3);
}

\newcommand{\ins}[2]{$\textsc{Ins}(#1,#2)$}
\newcommand{\del}[2]{$\textsc{Del}(#1,#2)$}

\tikzset{node distance = \vdist and \hdist}
\tikzset{path/.style = {draw, rounded corners, ultra thick, #1}}

\begin{tikzpicture}
  \uncover<1->{
    \statesplit{0}{\epsilon}{}
    \statesplit{1}{x}{below = of 0}
    \transition{0}{1}{1: \ins{x}{0}}

    \statesplit{12}{\epsilon}{below left = of 1}
    \statesplit{13}{ax}{below right = of 1}
    % \statesplit{123}{a}{below = 2.1*\vdist of 1}
    \statesplit{123}{a}{below right = of 12}
    \transition{1}{12}{2: \del{x}{0}}
    \transition{1}{13}{3: \ins{a}{0}}
    \transition{12}{123}{3: \ins{a}{0}}
    \transition{13}{123}{2: \del{x}{1}}

    % \statesplit{14}{xb}{right = 1.5*\vdist of 13}
    % \transition{1}{14}{4: \ins{b}{1}}

    % \statesplit{124}{b}{right = 2.1*\hdist of 123}
    \statesplit{124}{b}{below right = of 13}
    \transition[near end]{12}{124}{4: \ins{b}{0}}
    
    % \statesplit{1234}{ba}{below = 2*\vdist of 13}
    \statesplit{1234}{ba}{below right = of 123}
    \transition{124}{1234}{3: \ins{a}{1}}
    \transition{123}{1234}{4: \ins{b}{0}}

    \statesplit{14}{xb}{above right = of 124}
    \transition{1}{14}{4: \ins{b}{1}}
    \transition{14}{124}{2: \del{x}{0}}
  }

  \uncover<2->{
    \path[path = {red}] ($(0.north) + (135:20pt)$) node[left] 
	  (sc1) {$S, C_1$} -- ++(-90:2.5*\vdist) -- ++(-135:2.5*\vdist) -- ++(-45:5.2*\vdist); % -- ++(-50:2.6*\vdist);
    \path[path = {blue}] ($(0.north) + (100:20pt)$) node[above = 0pt] 
	  (c2) {$C_2$} -- ++(-90:2.7*\vdist) -- ++(-45:2.5*\vdist) -- ++(-135:2.2*\vdist) -- ++(-45:2.4*\vdist);
    \path[path = {teal}] ($(0.north) + (60:15pt)$) node[right] 
	  (c3) {$C_3$} -- ++(-90:2.5*\vdist) -- ++(-18:5.5*\vdist) -- ++(-135:5.0*\vdist);
  }
\end{tikzpicture}}

    \uncover<2->{每个replica的行为对应于该状态空间中的一条\hl{\red{路径}}}
  \end{center}
\end{frame}
%%%%%%%%%%%%%%%%%%%%

%%%%%%%%%%%%%%%%%%%%
\begin{frame}{}
  \centerline{\teal{\LARGE CJupiter 满足 Weak List Specification}}
\end{frame}
%%%%%%%%%%%%%%%%%%%%

%%%%%%%%%%%%%%%%%%%%
\begin{frame}{}
  \begin{center}
    {\large 关注某个$n$-ary有序状态空间, \hl{三步骤}证明\red{``状态对兼容性''}}
  \end{center}

  \fignocaption{width = 0.45\textwidth}{figs/cjupiter-allinone-path}
\end{frame}
%%%%%%%%%%%%%%%%%%%%

%%%%%%%%%%%%%%%%%%%%
\begin{frame}{}
  \centerline{\circled{1} 任取两个状态节点 $v_1$和$v_2$}

  \begin{clemma}[LCA (Lowest Common Ancestor)]
    $n$-ary 有序状态空间中的任意一对状态节点都有\red{唯一的}最近公共祖先。
  \end{clemma}

  \begin{columns}
    \column{0.60\textwidth}
      \fignocaption{width = 0.50\textwidth}{figs/lca}
      \column{0.30\textwidth}
	\[
	  v_0 = \text{LCA}(v_1, v_2)
	\]
  \end{columns}
\end{frame}
%%%%%%%%%%%%%%%%%%%%

%%%%%%%%%%%%%%%%%%%%
\begin{frame}{}
  \centerline{\circled{2} 考虑从 $v_0 = \text{LCA}(v_1, v_2)$ 到 $v_1$ 和 $v_2$ 的两条路径}

  \begin{clemma}[Disjoint Paths]
    路径 $P_{v_0 \leadsto v_1}$ 上包含的操作集 $O_{v_0 \leadsto v_1}$ 与路径 $P_{v_0 \leadsto v_2}$ 上包含的操作集
    $O_{v_0 \leadsto v_2}$ 不相交。
  \end{clemma}

  \begin{columns}
    \column{0.60\textwidth}
	\fignocaption{width = 0.50\textwidth}{figs/disjoint-paths}
      \column{0.30\textwidth}
	\[
	  v_0 = \text{LCA}(v_1, v_2)
	\]
  \end{columns}
\end{frame}
%%%%%%%%%%%%%%%%%%%%

%%%%%%%%%%%%%%%%%%%%
\begin{frame}{}
  \centerline{\circled{3} 考虑两条路径上的状态}

  \begin{clemma}[Compatible Paths]
    $P_{v_0 \leadsto v_1}$ 上的任一状态 $v$ 与 $P_{v_0 \leadsto v_2}$ 上的任一状态 $v'$ 是兼容的。
  \end{clemma}

  \begin{columns}
    \column{0.60\textwidth}
	\fignocaption{width = 0.55\textwidth}{figs/compatible-paths}
      \column{0.40\textwidth}
	\[
	  v_0 = \text{LCA}(v_1, v_2)
	\]

	\pause
	\vspace{0.50cm}
	\begin{center}
	  \hl{$\therefore$ $v_1$ 和 $v_2$ 是兼容的}
	\end{center}
  \end{columns}
\end{frame}
%%%%%%%%%%%%%%%%%%%%
